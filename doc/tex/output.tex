\documentclass[manual.tex]{subfiles} 
\begin{document}

\label{sec:output} 

SMAC outputs a variety of information to log files, trajectory files,
and state files. Most of the files are human readable, and this section
describes these files.
\\
\textsc{Note:} All output is written to the \textbf{outdir} in the \textbf{-~$\!$-rungroup} sub-directory.

%%%%%%%%%%%%%%%%%%%%%%%%%%%%%%%%%%%%%%%%%%%%%%%%%%%%%%%%%%%%%%%%%%%%
\subsection{Logging Output}
%%%%%%%%%%%%%%%%%%%%%%%%%%%%%%%%%%%%%%%%%%%%%%%%%%%%%%%%%%%%%%%%%%%%

	SMAC uses slf4j (\url{http://www.slf4j.org/}), a library that allows for abstracting and replacing the logging system with ease, and uses logback (\url{http://logback.qos.ch/}) as the default logging system. While there is limited ability to change logging options via the command line (\eg{, \textbf{-$~\!$-log-level}, \textbf{-$~\!$-console-log-level}, \textbf{-$~\!$-log-all-call-strings},\textbf{-$~\!$-log-all-process-output}}). It is possible that by setting a system property\footnote{You'd have to edit the start up script smac or smac.bat}  you can override the configuration by using \texttt{-Dlogback.configurationFile=/path/to/config.xml}

	
	\textsc{Note:} If you replace the logger in SMAC or modify the configuration file, the logging command line options may no longer work.


By default SMAC writes the following logging files out to disk (\textsc{Note:} The $N$ represents the \textbf{-~$\!$-seed} setting): 
\begin{description}
\item [{log-run$N$.txt}] A log file that contains a full dump of all the information logged, and where it was logged from.
\item [{log-warn$N$.txt}] Contains the same information as the above file,
except only from warning and higher level messages.
\item [{log-err$N$.txt}] Contains the same information as the above file,
except only from error messages.
%% Removed in 2.0.2
%% \item [{raw-run$N$.txt}] Only the actual logging messages (this is easier to grep for needed information).
%% \item [{runhashes-run$N$.txt}] A file that contains only the Run Hash Codes for a given run see the corresponding entry in the \textbf{FAQ}.

\end{description}


%%%%%%%%%%%%%%%%%%%%%%%%%%%%%%%%%%%%%%%%%%%%%%%%%%%%%%%%%%%%%%%%%%%%
\subsubsection{Interpreting the Log File}
%%%%%%%%%%%%%%%%%%%%%%%%%%%%%%%%%%%%%%%%%%%%%%%%%%%%%%%%%%%%%%%%%%%%

SMAC basically goes through three phases when executing:
\begin{itemize}
\item Setup Phase Input files are read, and their arguments validated. Everything necessary to execute the Automatic Configuration Phase is constructed. This phase ends, once the following message appears:
\scriptsize{
\begin{verbatim}
SMAC started at: 10-Apr-2014 10:01:40 AM. Minimizing penalized average runtime (PAR10)
\end{verbatim}
}
\normalsize 
\item Automatic Configuration Phase: SMAC is now actively configuring the target algorithm. SMAC will spend most of it's time here, and outputs it's progress. 

There are two types of messages you will see here:


\scriptsize{
\begin{verbatim}
1) Incumbent changed to: config 2 (internal ID: 0x7), with penalized average runtime (PAR10): 7.1;
 estimate based on 2 runs.
 Sample call for new incumbent config 2 (internal ID: 0x7): 
cd saps; ruby saps_wrapper.rb instances/train/SWlin2006.19724.cnf 0 10.0 2147483647 -1
 -alpha '1.1' -ps '0.1' -rho '0.84' -wp '0.06'  

\end{verbatim}
}
\normalsize

This signifies that the incumbent (the best configuration found so far), has been changed to configuration 2 (this ID is used in some files that SMAC will output). It also gives a sample estimate of the performance of this configuration on the instances we have seen already.

Next a sample call is given for this configuration so that you can test it yourself. From this you can determine the actual configuration selected, in this example it is:

\begin{center}
\emph{-alpha '1.1' -ps '0.1' -rho '0.84' -wp '0.06'}
\end{center}


\scriptsize{
\begin{verbatim}
2) Updated estimated penalized average runtime (PAR10) of the same incumbent: 3.86;
estimate now based on 4 runs.
\end{verbatim}
}
\normalsize

As SMAC continues to run it will continue refining the estimate by making more samples of the incumbents configuration, and it will occasionally provide you with an update. This number can vary wildly as SMAC learns more about your instance distribution.



\item 
Once SMAC is completed, it will output some summary statistics:

\scriptsize{
\begin{verbatim}
=======================================================================================
SMAC has finished. Reason: total CPU time limit (600.0 s) has been reached.   
SMAC's final incumbent: config 45 (internal ID: 0xBB), 
 with estimated penalized average runtime (PAR10): 0.12 ,
 based on 5 run(s) on 5 training instance(s).
Total number of runs performed: 171, total CPU time used: 604 s,
 total wallclock time used: 607 s, total configurations tried: 113.
=======================================================================================
\end{verbatim}
}
\normalsize
The first line indicates why SMAC terminated, in this case the CPU time limit
was exceeded. It provides an updated estimate of the objective, in this case 0.12. 

\item Offline Validation Phase, depending on the options used this can also take a large fraction of SMAC's runtime. The logic here is actually quite simple, as it largely only requires running many algorithm runs and computing the objectives from them.

\scriptsize{
\begin{verbatim}
-------------------------------------------------------
Minimized penalized average runtime (PAR10):
Final time: 607 config 45 (internal ID: 0xBB):  0.14 on the test set

Sample call for the final incumbent:
cd saps; ruby saps_wrapper.rb instances/train/SWlin2006.19724.cnf 0 10.0 2147483647 -1
 -alpha '1.1' -ps '0.1' -rho '0.84' -wp '0.06'

Additional information about run 1 in: smac-output/run1/
-------------------------------------------------------
\end{verbatim}
}
\normalsize

\end{itemize}

%%%%%%%%%%%%%%%%%%%%%%%%%%%%%%%%%%%%%%%%%%%%%%%%%%%%%%%%%%%%%%%%%%%%
\subsection{State Files}\label{subsec:state-files}
%%%%%%%%%%%%%%%%%%%%%%%%%%%%%%%%%%%%%%%%%%%%%%%%%%%%%%%%%%%%%%%%%%%%

State files allow you to examine and potentially restore the state of SMAC at a specific point of it's execution. The files are written to the state-run$N$/ sub-directory, where $N$ is the value of \textbf{-~$\!$-seed} option. 


All files have the following convention as a suffix either \texttt{it} or \texttt{CRASH} followed by either the iteration number $M$, or in some cases \texttt{quick} or \texttt{quick-bak}.

The state is saved for every iteration $m$, where $m=2^{n}$ $n \in \mathbb{N}$, additionally it is saved when SMAC completes whether successfully or due to crash.
% Finally the \texttt{java\_obj\_dump} file is written every iteration, though only the last two are saved (the \texttt{quick} file stores the present iteration, and \texttt{quick-bak} stores the previous iteration} . 

The following files are saved in this state directory (ignoring the suffix):

\begin{description}
\item[java\_obj\_dump-v2-it$M$.obj] Stores (Java) serialized versions of the the incumbent and the random object state. In general there is no need to look at this file, and it is not human readable.

\item[paramstrings-it$M$.txt]	  Stores a human readable setting of each configuration ran, with a prefix of the numeric id of the configuration (as used in the logs, and other state files).

\item[uniq\_configurations-$M$.csv] Stores the configurations ran in a more concise but effectively un-human readable form. The first column again is the numeric id of the configuration (as used in the logs, and other state files).

\item[runs\_and\_results-it$M$.csv] Stores the result of every run of the target algorithm that SMAC has done. The first 13 columns (after the header row are designed to be backwards compatible with SMAC versions 1.xx. Each column is labelled with what data it contains, the following columns deserve some description.
	\begin{enumerate}
	\item[\texttt{Instance ID}]	This is the instance used, and is the $n^{th}$ \textbf{Instance Name} specified in the \textbf{instance\_file} option.
	\item[\texttt{Response Value(y)}]  This is the value determined by the \textbf{run\_obj} on the run.
	\item[\texttt{Censored}] Indicates whether the \texttt{Cutoff Time Used} field is less than the \textbf{cutoff\_time} in the original run. 0 means \texttt{false}, 1 means \texttt{true}.
	\item[\texttt{Run Result Code}]	This is a mapping from the \texttt{Run Result} to an integer for use with previous versions.
	\end{enumerate}
	
	\item[param-file] If \textbf{-$~\!$-save-context} is enabled, a copy of the \textbf{paramfile} will be in the state folder 
	
	\item[instances] If \textbf{-$~\!$-save-context} is enabled, a copy of the \textbf{instance\_file} will be in the state folder 
	
	\item[instance-features] If \textbf{-$~\!$-save-context} is enabled, and SMAC is running with features, then a copy of the \textbf{feature\_file} will be in the state folder.

	\item[scenario] If \textbf{-$~\!$-save-context} is enabled, and SMAC is using a scenario file, then a copy of the  \textbf{-$~\!$-scenario-file} will be in the state folder.

\end{description}

%%%%%%%%%%%%%%%%%%%%%%%%%%%%%%%%%%%%%%%%%%%%%%%%%%%%%%%%%%%%%%%%%%%%
\subsection{Trajectory File}
%%%%%%%%%%%%%%%%%%%%%%%%%%%%%%%%%%%%%%%%%%%%%%%%%%%%%%%%%%%%%%%%%%%%

SMAC also outputs a trajectory file into the \texttt{detailed-traj-run-$N$.csv}  and outline the incumbent (by id) over the course of execution and it's performance. The first line gives the 
\textbf{-~$\!\!$-rungroup}, and then the  \textbf{-~$\!\!$-seed}.

The rest of the file follows the following format:

\begin{tabular}{|c|c|}
\hline 
Column Name & Description\tabularnewline
\hline 
\hline 
CPU Time Used & Sum of all execution times \\ & and CPU time of SMAC\tabularnewline
\hline 
Estimated Training Performance & Performance of the Incumbent under the given \\ &  \textbf{-~$\!\!$-run-obj
}and \textbf{-~$\!\!$-overall-obj}\tabularnewline
\hline 
Wallclock Time & Time of entry with respect to wallclock time. %Standard deviation of the Incumbent under the given \textbf{--runObjective }and \textbf{--overallObjective}
\tabularnewline
\hline 
Incumbent ID & The ID of the incumbent \\ & as listed in the \textbf{param\_strings} file in \textsection \ref{subsec:state-files}, and the logs
\tabularnewline
\hline 
Automatic Configurator (CPU) Time & CPU Time used of SMAC\tabularnewline
\hline 
Full Configuration & The full configuration of this incumbent \tabularnewline
\hline 
\end{tabular}
\\
\textsc{Note}: SMAC also outputs \texttt{traj-run-$N$.txt} the first five columns are the same, but the remaining columns represent the configuration, with each cell being a key and value. This is identical the trajectory file outputted by ParamILS.

%%%%%%%%%%%%%%%%%%%%%%%%%%%%%%%%%%%%%%%%%%%%%%%%%%%%%%%%%%%%%%%%%%%%
\subsection{Validation Output}
%%%%%%%%%%%%%%%%%%%%%%%%%%%%%%%%%%%%%%%%%%%%%%%%%%%%%%%%%%%%%%%%%%%%

When Validation is completed four files are outputted, (again $N$ is the value of the \textbf{-~$\!\!$-seed} argument). Finally depending on which options are used the, especially with \texttt{smac-validate}, the actual file name outputted may vary.

\begin{enumerate}
\item \texttt{validationResults-traj-run-$N$-walltime.csv}:
A CSV file that contains a summary of the results of validation, the \textit{Validation Configuration ID} maps to a line in the next file

\item \texttt{validationCallStrings-traj-run-$N$-walltime.csv}:
A CSV file containing a mapping between \textit{Validation Configuration ID} to
the actual configuration, and a sample call string.

\item \texttt{validationPerformanceDebug-traj-run-$N$-walltime.csv}:
A CSV file that contains a detailed breakdown of how the final validation score
was obtained. This file is meant for human consumption, and not for parsing.

\item \texttt{validationObjectiveMatrix-traj-run-$N$-walltime.csv}:
A CSV file that contains a table, for each row (configuration) the objective for the problem instance seed pair as given in the column.

\item \texttt{validationRunMatrix-traj-run-$N$-walltime.csv}:
A CSV file that contains a table, for each row (configuration) the response from the wrapper (ignoring the prefix).

\end{enumerate}

%%%%%%%%%%%%%%%%%%%%%%%%%%%%%%%%%%%%%%%%%%%%%%%%%%%%%%%%%%%%%%%%%%%%
\subsection{JSON Output}
%%%%%%%%%%%%%%%%%%%%%%%%%%%%%%%%%%%%%%%%%%%%%%%%%%%%%%%%%%%%%%%%%%%%

For each run of SMAC, SMAC will also output a file \texttt{live-rundata-N.json}. The format of the file consists of an array representation of all of the problem instances. Then the individual runs are reported one after the other. The actual JSON representation is meant to be a mostly semantic view of the objects and is enough to construct all of the runs data. 

%
%
%\item \texttt{rawValidationExecutionResults-run$N$.csv}:
%
% CSV File containing a list of the configuration, 
%seeds \& instance run and the corresponding result and the result of the target algorithm execution. This file is mainly for debugging.
%
%\item 
%\texttt{validationInstanceSeedResult-run$N$.csv}:
%
% CSV File containing a list of seeds \&
%instances and the resulting response value. Again this file is mainly for debugging, but is easier to parse than the previous.
%\item 
%\texttt{validationResultsMatrix-run$N$.csv}:
%
%CSV File containing the list of instances
%on each line, the next column is the aggregation of the remaining columns under the 
%\textbf{overall\_obj}. Finally there is one additional row that
%gives the aggregation of all the individual \textbf{overall\_obj},
%aggregated in the same way.
%
%\texttt{validationResults-run$N$.csv}
%\\
%CSV File containing the result of the validation. The columns are defined as follows:
%
%\begin{tabular}{|c|c|}
%\hline 
%Column Name & Description\tabularnewline
%\hline 
%\hline 
%Tuner Time &  The tuner time when validation occurred \tabularnewline
%\hline 
%Emperical Performance & The incumbent's performance on the training set \tabularnewline
%\hline 
%Test Set Performance & The incumbent's performance on the test set \tabularnewline
%\hline
%AC Overhead Time & Total CPU Time Used by the Automatic Configurator  \tabularnewline
%\hline 
%Sample Call String \tabularnewline
%\hline

%\end{tabular}




%\end{enumerate}


\end{document}